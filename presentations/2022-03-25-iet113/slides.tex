\documentclass[aspectratio=169]{beamer}
\usetheme{Boadilla} % plainest one with slide number footer

% generic packages

\usepackage[utf8]{inputenc}
\usepackage[english]{babel}
\usepackage{amsmath}
\usepackage{multicol}
\usepackage{ulem}

% about the presentation
\title{CoAP: Non-traditional response forms}
\hypersetup{pdftitle={CoAP: Non-traditional response forms}}
\subtitle{\texttt{draft-bormann-core-responses-01}}
\author{Carsten~Bormann, \textit{Christian~Amsüss}}
\date{2022-03-25 \\CoRE at IETF 113 in Vienna}

% attach self

\usepackage{embedfile}
\embedfile{\jobname.tex}

\begin{document}

\frame{\titlepage}


\newcommand{\otheritems}{
% 2010 
\item[RFC7252 multicast] Single request, different responses with different source addresses

% 2010
\item[RFC7641] Observe notifications come in until further notice

% 2011
\item[core-coap-sms] Response-To-Uri-Host/-Port -- triangular requests

% 2013
\item[core-coap-endpoint-id] Observations across server address changes

% 2017
\item[core-responses-00] Configured requests, Response-For

% 2020
\item[RFC9177] Responding with multiple blocks

% 2020
\item[core-groupcomm-proxy] Multicast-Timeout
}

\begin{frame}{A history of \ldots{} relevant things}\large
	\begin{description}[RFC7252 simplest case]
		% 2010 
		\item[RFC7252 simplest case] Single request, single response\footnote{Retransmits are not necessarily identical, but either way the client usually only processes one.}

		\otheritems
	\end{description}

	\raggedleft

	\ldots all on the same token.
\end{frame}

\begin{frame}{A history of non-traditional responses}\large
	\begin{description}[RFC7252 simplest case]
		% 2010 
		\item[\color{gray}RFC7252 simplest case] \color{gray} Single request, single response\footnote{The retransmits, even when changed, are barely not considered nontraditional.}\color{black}

		\otheritems
	\end{description}

	\raggedleft

	\ldots which can all be described with common terminology.
\end{frame}

\begin{frame}{Defining non-traditional responses}\Large
	\textit{Non-traditional response}: A response that is not the single response generated for a request received on the same transport.

	\bigskip

	One response is non-traditional $\Rightarrow$ all responses are non-traditional \ldots{} but often, only one is ``matching''.

	\pause\bigskip

	\textit{Non-matching response}: A response that has properties (typically options) that make it incompatible with the original request, and thus in particular unsuitable as a cached response to that request (but possibly suitable to populate the cache for a similar request).
\end{frame}

\begin{frame}{What to do with it?}\large
	\begin{itemize}
		\item Provide terminology and general concepts usable for other specifications.

			\qquad groupcomm-proxy and multicast-notifications are concrete candidates.
		\item Provide terminology and general concepts usable for implementers.

			\qquad Ongoing experimentation in Python and Rust implementations.
		\item Provide concepts usable for applications.

			\qquad Statically configured requests.

			\qquad (Dynamic configured requests can follow multicast-notifications pattern.)

		\item Provide options usable for generic applications (Response-For, Respond-To, Leisure-Time).

			\qquad Can double as examples for further specifications.
	\end{itemize}
\end{frame}

\begin{frame}{Next steps}\Large 
	\begin{itemize}
		\item Is the working group interested in general work on non-traditional responses?
		\item Is the working group interested in this document for\ldots
			\begin{itemize}
				\item \Large \ldots its taxonomy?
				\item \Large \ldots its concrete options?
			\end{itemize}
		\item Are there any conflicts with existing concepts?
		\item Any other feedback?
	\end{itemize}

\end{frame}

% TBD plug taps?

\end{document}
